\section*{Abstract}
\selectlanguage{ngerman}
Unser Partner, Ludwig System, entwickelt Traversen zur Balancierung schwerer Lasten wie Wände und Dächer. 
Die Befestigung dieser Lasten erfolgt bisher manuell, was eine gefährliche und zeitintensive Aufgabe darstellt. 
Zur Automatisierung dieses Prozesses sollte ein System entwickelt werden, das fähig ist, Anschlagspunkte automatisch erkennen zu können. 
In diesem Dokument wird ein Prototyp vorgstellt, der mittels AprilTags Anschlagspunkte identifiziert.
Dies wird durch die Posenschätzung der Marker erreicht und durch die Anordnung der Marker kann die Position und Rotation des Anschlagspunkt in Echtzeit berechnet werden und Ausgegeben werden.
Die Implementierung konnte die meisten Anforderungen von Kunden erfüllen, ausser die Genauigkeiten zur Neigung, da keine Tests durchgeführt werden konnten.
Weiters konnte nicht die Zuverlässigkeit des Systems bei Wetter evaluiert werden. 
Zurzeit ist die Latenz in der Implementation zu hoch. 
\vspace{2ex}

\textbf{Keywords:}

Fiducial Marker, Apriltags, ArUco, 3D Lokalisierung, OpenCV, Python

\clearpage


