\section{Konzept}


\subsection{Überblick}
Ludwig Systems benötigt eine Lösung, welche von der Traverse aus die Ist-Koordinaten, die Rotation und Neigung der Last ausgibt. Das Konzept soll dieses Ziel erfüllen um in der Zukunft die Traverse automatisch bewegen zu können. Des Weiteren soll das Konzept sicherstellen, dass die Genauigkeiten ,welche vom Kunden gegeben wurde, eingehalten werden können. 
Da sich die Art von Last ändern kann, wurde der Fokus auf die Lokalisierung der Anschlagspunkte gelegt. 
Es wurden 2 Konzepte ausgearbeitet, welche diese Ziele erfüllen können. Beim ersten Konzept wird ein Machine-Learning Modell benutzt um die Anschlagspunkte zu erkennen und 3D Lokalisierung durchzuführen und beim zweiten Konzept werden die Anschlagspunkte durch AR Marker 
Lokalisiert.

\subsection{Anforderungen}\label{requirements}


\subsection{Konzept: Lokalisierung durch Machine Learning}

In diesem Konzept geht es darum die Anchlagspunkte mithilfe von Objekterkennung durch YOLO zu erkennen und die Ist-Koordinaten durch 3D Lokalisierung zu erhalten. 
Für dieses Konzept wird ein Machine Learning Modell benötigt, welches diese Aufgabe erfüllen kann. 
Weiters würden gelabelte Bilder von den Lasten und Anschlagspunkte zum Lernen und weitere Bilder zum Testen benötigt.

\subsection{Konzept: Lokalisierung durch AR Marker}

In diesem Konzept geht es darum die Anschlagspunkte durch AR Marker, wie ArUco oder AprilTags, zu lokalisieren. Dabei werden die Marker um den Anschlagspunkt angeordnet. Wenn die Marker dann erkannt werden, kann durch Posenschätzung die 3D Koordinaten und 3D Rotationen berechnet werden. Durch die Anordnung der Marker kann dann die 3D Koordinaten vom Anschlagspunkt berechnet werden. 
Für dieses Konzept werden Sticker von den Markern benötigt, welche eine genaue Grösse haben. 

\subsubsection{Marker Anordnung}

Wie in der Abbildung (ref) zu sehen ist werden 4 Marker in einem Kreuz-Muster um den Anschlagspunkt angeordnet. Für Anschlagspunkt 1 werden Marker mit den IDs von 0 bis 3 benutzt und in der Reihenfolge links, oben, rechts und dann unten angeordnet. Für Anschlagspunkt 2 Marker mit den IDs von 4 bis 7 benutzt. Dadurch können zwischen den zwei Anschlagspunkten differenziert werden wenn beide Anschlagspunkte im Bild sind.Falls weitere Anschlagspunkte benötigt werden, sollten die Marker-IDs-Reihenfolge und Position fortgeführt werden. Diese Anordnung sorgt dafür dass mindestens ein Marker erkannt werden kann, soweit die Traverse in der definierten Grenze ist. 
(redundanz erhöht stabilität)

\subsubsection{Marker Grösse}

Die Grösse eines Markers hat einen direkten Einfluss auf die Genauigkeit der Posenschätzung. Damit die Genauigkeiten sichergestellt werden können, muss die grösst mögliche Seitenlänge *l* eines Markers in cm berechnet werden in der Kreuz-Muster Anordenung, welche im Kapitel (ref) beschrieben ist.

(Rechnung von l)


\subsubsection{Mittelpunkt-Berechnung der Marker Anordnung}

Um von den 3D Koordinaten der Markern zu den 3D Koordinaten vom Anschlagspunkt zu kommen, muss eine Translation gemacht werden. Da die Berechnung zum Mittelpunkt auch mit nur einem Marker funktionieren muss, wird von jede Marker-Koordinaten in die Mitte des Keuz-Musters translatiert. Die Anordung der Marker und die damit zusammenhängende Marker-ID bestimmt welche Translation durchgeführt werden muss. Marker mit der ID 0 und 4 werden um + l und Marker mit ID 2 und 6 um -l in der X-Achse translatiert. Marker mit der ID 1 und 5 werden um -l und Marker mit ID 3 und 7 um +l in der Y-Achse translatiert. Falls mehrere Marker erkannt werden, wird der Mittelwert aller neuen Punkten genommen.

\subsection{Evaluation Konzepte}
Um die Lokalisierung durch Machine-Learning durchzuführen braucht es genügend gelabelte Bilder um ein Modell zu trainieren, da kein vortrainiertes Modell vorhanden ist. Das auftreiben und labeln von diesen Bildern würde Zeit und Kosten brauchen. Die Kunden bestätigten das Marker an der Last angebracht werden können und dass sichergestellt werden kann dass diese den gleichen Abstand zueinander haben, weshalb das zweite Konzept ausgewählt wurde.

\subsection{Kamera Eigenschaften}

Um die Anforderungen vom\ref{requirements} zu erfüllen braucht das System eine Kamera, welche sicherstellt dass die Marker genau erkannt werden können.

Dafür ausschlaggebend sind die Markergrösse und eine hohe Bildauflösung\cite{noauthor_designing_2020}. 

Ein weiteres Kriterium für eine genauere Posenschätzung ist die FOV, also Field of View. 
Denn bei höheren FOVs kommt es zu verzerrungen, was dazu resultiert, dass Marker, die weiter von der Bildmitte entfernt sind, keine geraden Linien haben, welches die Posenschätzung ungenauer macht.


\subsubsection{Kamera horizontale Bildwinkel}

Die Kamera sollte von der Mindesthöhe, also 75cm, beide Anschlagspunkte sehen können.

\begin{figure}[H]
    \centering
    \includegraphics[width=0.5\textwidth]{graphics/KameraFOV.png}\hfill%
    \caption{Kameras decken die Last komplett ab}
    \label{fig:FOV}
\end{figure}

Um einen grossen horizontalen BildWinkel zu vermeiden, wird wie in der Abbildung\ref{fig:FOV} zwei Kameras verwendet. 
Beide Kameras sollten für je ein Anschlagspunkt verantwortlich sein und so im ideal Fall, wenn die Traverse direkt über der Last ist, mindestens von der Mitte der Last bis zum Anschlagspunkt reichen.

Für die Berechnung des horizontalen Bildwinkels werden folgende Variablen benutzt:

Mindesthöhe = h = 75cm\\
Halb-Distanz Mitte-Anschlagspunkt = d = 117.5cm\\

Damit kann man den Bildwinkel berechnen welches beide Kameras benutzen müssen:

\begin{equation}
    Bildwinkel = f = 2*\arctan\frac{d}{h} = 2*\arctan\frac{117.5}{75} = 114.9°
    \label{eq:FOV}
\end{equation}


\subsubsection{Kamera Bildauflösung}
Um festzustellen welche Bildauflösung optimal ist, muss die Genauigkeit der Positionierung berechnet werden.

Mit den benötigten Variablen:

 Marker Grösse = l = 6.3cm\\
 Anz. Bits in Länge = b\\
 horizontale FOV = f = 114.9°\\
 horizontale Bildauflösung = r\\
 Anz Pixel pro Bit = p\\
 Max Distanz = m = 200cm\\

b ist durch den Marker bestimmt und besagt wie viele Bits in einer Markerbreite sind. 
Für folgende Rechnungen wird der Apriltag 41h12 benutzt welcher 9 Bits hat, also b = 9.

Damit die Genauigkeit der Positionierung berechnet werden kann, muss man erst bestimmen auf wie viele Pixel genau die Lokalisierung ist.

Dafür benutzt man die Formel\ref{eq:maxDistance}, welche die Maximal Distanz ausrechnet\cite{noauthor_designing_2020}:
\begin{equation}
m = \frac{t}{2\tan\frac{b*f*p}{2r}}.
\label{eq:maxDistance}
\end{equation}

und löst die Formel\ref{eq:maxDistance} nach p auf:
\begin{equation}
p = \frac{2r*\arctan\frac{t}{2*m}}{b*f} = \frac{2r*\arctan\frac{6.3cm}{400cm}}{9*114.9°}
\label{eq:PixelperBits}
\end{equation}

Danach muss die Grösse eines Bildpixels in der Realität berechnet werden.

\begin{equation}
g = \frac{313.334cm}{r}
\label{eq:PixelReal}
\end{equation}

Mit p und g aus der Formel \ref{eq:PixelReal} kann die Genauigkeit pro Meter berechnet werden.

\begin{equation}
q = g * p
\label{eq:precision}
\end{equation}

Wenn man jetzt diese Formeln einsetzt für verschiedene Bildauflösungen.

\begin{center}
    \begin{tabular}{ c c c }
    \label{tab:resolutions}
     Bildauflösung & p & q \\ 
     1920x1080 & 3.35 & 0.547 \\  
     3840x2160 & 6.701 & 0.547 \\
     4096x2160 & 7.1481 & 0.5468 \\ 
\end{tabular}
\end{center}

Wie man sieht ist die Genauigkeit  bei allen gleich.
