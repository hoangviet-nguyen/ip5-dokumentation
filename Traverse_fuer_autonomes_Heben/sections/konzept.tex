\section{Konzept}

Ludwig Systems benötigt eine Lösung, welche von der Traverse aus die Ist-Koordinaten, die Rotation und Neigung der Last ausgibt. Das Konzept soll dieses Ziel um in der Zukunft die Traverse automatisch bewegen zu können. Des Weiteren soll das Konzept sicherstellen, dass die Genauigkeiten ,welche vom Kunden gegeben wurde, eingehalten werden können.

\subsection{Konzept 1: Objekterkennung durch Machine Learning}

\subsection{Konzept 2:  Posenschätzung durch AR Marker}

\subsection{Evaluation Konzepte}

\subsection{Intrinsische Kalibrierung der Kamera}
(Erklärung wie die Kalibrierung funktioniert )

\subsection{Lokalisierung Last}
(Erklärung wie die Lokalisierung mithilfe von Marker funktionieren sollte.)
\subsubsection{Marker Anordnung}
(Erklärung Anordnung und wieso)
\subsubsection{Marker Grösse}
(Erklärung der Rechnung für Marker Grösse)
\subsubsection{Mittelpunkt-Berechnung der Marker Anordnung}
(Erklärung der Rechnung um vom Diamand-Marker-Muster auf die Mitte) 

\subsection{Kamera Spezifikationen}
(Erklärung welche Kamera Eigenschaften gebraucht werden und wieso wir diese brauchen.)
\subsubsection{Kamera Position}
(Erklärung wo Kameras Positioniert sind und Rechnung von horizontale FOV)
\subsubsection{Kamera Eigenschaften}
(Erklärung der Rechnung horizontale Bildauflösung und auflistung aller Eigenschaften ausgehend der Marker Grösse.)
