\section{Einleitung}

Der Einsatz von Kränen ist auf Baustellen unerlässlich, um schwere Lasten zu bewegen. Aktuell erfordert das Anhängen dieser Lasten an den Kran manuelle Arbeit durch Bauarbeiter. Dies ist besonders zeitaufwendig, wenn der Schwerpunkt der Last ungleichmäßig verteilt ist und führt häufig dazu, dass die Last mehrmals abgesetzt und angehoben werden muss, um sie korrekt auszurichten. Dies verursacht erhebliche Wartezeiten für andere Teammitglieder. Die Firma Ludwig System hat eine automatische Traverse entwickelt, die das Ausrichten und Positionieren der Last erleichtert, jedoch muss auch mit dieser aktuellen Version ein Mitarbeiter die Last manuell an die Traverse anhängen.

Die nächste Generation der Traverse zielt darauf ab, den Prozess der Lastenaufhängung zu automatisieren. Ludwig System arbeitet derzeit intensiv an dieser Entwicklung und hat bereits Prototypen eines Greifarms erstellt, der dazu bestimmt ist, die Anschlagspunkte automatisch zu heben. Ein entscheidender Schritt in diesem Prozess ist die präzise Berechnung der Position dieser Anschlagspunkte.

Auf den ersten Blick erscheinen moderne Lösungen aus dem Bereich des maschinellen Lernens als ideal für diese Aufgabe. Jedoch besteht die Herausforderung darin, dass für das Training eines effektiven Modells eine umfangreiche Menge an Daten benötigt wird, die derzeit noch nicht ausreichend verfügbar ist. Zusätzlich erfordert die Anwendung solcher Technologien weiterführende Forschungen, um aus den erkannten Anschlagspunkten die Entfernung und Ausrichtung relativ zur Kamera zu bestimmen. Diese Berechnungen werden noch komplexer, wenn die Anschlagspunkte in verschiedenen Dimensionen vorliegen, was eine zusätzliche Anpassung und Verfeinerung der Algorithmen erfordert

Ein weiterer Ansatz ist die Verwendung von Passmarkern. Diese Marker haben den Vorteil, dass sie in standardisierten Größen verfügbar sind, was ihre Erkennung erleichtert \cite{astrobee2023}. Durch die Analyse der Eckpunkte eines Markers kann die Position relativ zur Kamera bestimmt werden \cite{localizationSystem}. Zusätzlich ermöglicht die Kodierung einer eindeutigen ID auf dem Marker die Feststellung seiner Ausrichtung. Theoretisch erlaubt dies dem Kranführer, die Traverse präzise auszurichten. Der Einsatz solcher Markierungssysteme wird bereits in der Robotik erfolgreich zur Positionsbestimmung von Objekten genutzt, was deren Effektivität und Zuverlässigkeit unterstreicht \cite{localizationSystem}.

Um die Marker effektiv erkennen zu können, spielen sowohl die Position als auch die Hardware der Kamera eine entscheidende Rolle. In Sektion 2 befassen wir uns damit, wie wir die Kamera strategisch an der Traverse positionieren können, um eine möglichst umfassende Sicht auf die Last zu gewährleisten. Grundsätzlich gilt, dass eine höherwertige Kamera zu präziseren Ergebnissen führt. Später präsentieren wir die Mindestanforderungen an die Kamera und erläutern, wie diese theoretisch berechnet wurden. Zusätzlich müssen die intrinsischen Spezifikationen der Kamera berücksichtigt werden, um sicherzustellen, dass die Marker aus der gewünschten Distanz effektiv erkannt werden können.


(Abschließend gilt es zu klären, inwieweit unser System gegenüber leichten Wetterbedingungen wie Regen, Schnee und Schatten robust ist. Diese Faktoren sind auf Baustellen allgegenwärtig und es ist entscheidend, dass das System auch unter solchen Bedingungen zuverlässig funktioniert und gegenüber Umgebungsstörungen unempfindlich bleibt.) Dieser Abschnitt muss noch die Frage gegenüber wetter resistenz beantworten